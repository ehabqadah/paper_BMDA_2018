

\subsubsection*{Pattern prediction over event streams}

\par The task of forecasting over time-evolving streams of data can be formulated in various different ways and with varying assumptions.
One of the most usual ways to formalize this task is to assume that the stream is a time-series of numerical values and the goal is to forecast at each time point $n$ the values at some future points $n+1$, $n+2$, etc., (or even the output of some function of future values). 
This is the task of time-series forecasting \cite{montgomery_introduction_2015}.
Another way to formalize this task is to view streams as sequences of events,
i.e., tuples with multiple, possibly categorical, attributes, like \textit{event type}, \textit{timestamp}, etc., and the goal is to predict future events or patterns of events.  In the work that we present here, we focus on this latter definition of forecasting (event pattern forecasting).  

\par A substantial body of work on event forecasting comes from the field of temporal pattern mining where events are defined as 2-tuples of the form $(\mathit{EventType},\mathit{Timestamp})$.
The ultimate goal is to extract patterns of events in the form either of association rules \cite{agrawal_mining_1993} or frequent episode rules \cite{mannila_discovery_1997}. 
These methods have been extended in order to be able to learn not only rules for detecting event patterns but also rules for predicting events.
For example, in \cite{vilalta_predicting_2002}, a variant of association rule mining is where the goal is to extract sets of event types that frequently lead to a rare, target event within a temporal window. 
\par In \cite{laxman_stream_2008}, a probabilistic model is presented
in order to calculate the probability of the immediately next event in the stream. 
This is achieved by using standard frequent episode discovery algorithms and combining them with Hidden Markov Models and mixture models.
The framework of episode rules is employed in \cite{fahed_efficient_2014} as well.
The output of the proposed algorithms is a set of predictive rules whose antecedent is minimal (in number of events) and temporally distant from the consequent.
In \cite{zhou_pattern_2015} a set of algorithms is proposed that target batch, online mining of sequential patterns, without maintaining exact frequency counts.
As the stream is consumed, the learned patterns can be used to test whether a prefix matches the last events seen in the stream, indicating a possibility of occurrence for events that belong to the suffix of the rule.

\par Event forecasting has also attracted some attention from the filed of Complex Event Processing (see \cite{Cugola:2012:PFI:2187671.2187677} for a review of Complex Event Processing).
One such early approach is presented in \cite{muthusamy_predictive_2010}.
Complex event patterns are converted to automata and subsequently
Markov chains are used in order to estimate when a pattern is expected to be fully matched.
A similar approach is presented in \cite{alevizos2017event},
where again automata and Markov chains are employed in order to provide (future) time intervals during which a match is expected with a probability above a confidence threshold. 